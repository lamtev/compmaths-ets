\documentclass[a4paper,14pt]{extarticle}
\usepackage[utf8x]{inputenc}
\usepackage[T1,T2A]{fontenc}
\usepackage[russian]{babel}
\usepackage{hyperref}
\usepackage{indentfirst}
\usepackage{listings}
\usepackage{color}
\usepackage{here}
\usepackage{array}
\usepackage{multirow}
\usepackage{graphicx}
\usepackage{caption}
\usepackage{subcaption}
\usepackage{chngcntr}
\usepackage{amsmath}
\usepackage{pgfplots}
\usepackage{pgfplotstable}
\counterwithin{figure}{section}
\counterwithin{equation}{section}
\counterwithin{table}{section}
\usepackage{tabularx}

%% Поля подписи и даты
\newcommand{\sign}[1][5cm]{%
\makebox[#1]{\hrulefill}
}

\usepackage[left=2cm,right=2cm,
top=2cm,bottom=2cm,bindingoffset=0cm]{geometry}


\begin{document}	% начало документа

\begin{titlepage}	% начало титульной страницы

	\begin{center}		% выравнивание по центру

		\large Санкт-Петербургский Политехнический Университет Петра Великого\\
		\large Институт компьютерных наук и технологий \\
		\large Кафедра компьютерных систем и программных технологий\\[4cm]
		% название института, затем отступ 6см
		
		 \huge Вычислительная математика\\[0.3cm] % название работы, затем отступ 0,5см
		 \large Расчётное задание №4\\[0.1cm]
		 \large Решение систем дифференциальных и разностных уравнений II порядка\\[8cm]

	\end{center}


	\begin{flushright} % выравнивание по правому краю
		\begin{minipage}{0.35\textwidth} % врезка в половину ширины текста
			\begin{flushleft} % выровнять её содержимое по левому краю

				\large\textbf{Работу выполнил:}\\
				\large Ламтев А.Ю.\\
				\large {Группа:} 23501/4\\
				
				\large \textbf{Преподаватель:}\\
				\large Цыган В.Н.

			\end{flushleft}
		\end{minipage}
	\end{flushright}
	
	\vfill % заполнить всё доступное ниже пространство

	\begin{center}
	\large Санкт-Петербург\\
	\large \the\year % вывести дату
	\end{center} % закончить выравнивание по центру

\thispagestyle{empty} % не нумеровать страницу
\end{titlepage} % конец титульной страницы

\vfill % заполнить всё доступное ниже пространство


\section{Задание}

\begin{enumerate}

\item Решить разностное уравнение

\begin{displaymath}
Z[n+2] - 9 \cdot Z[n+1] + 18 \cdot Z[n] = n + 2
\end{displaymath}
\begin{displaymath}
Z(0) = 1
\end{displaymath}
\begin{displaymath}
 Z(1) = 3
\end{displaymath}

\item Осуществить проверку решения для $n = 2$

\end{enumerate}

\section{Решение}

\subsection{Методом неопределённых коэффициентов}

$Z[n] = Z^{o}[n] + Z^{\text{н}}[n]$

$Z[n+2] - 9 \cdot Z[n+1] + 18 \cdot Z[n] = 0$

$Z[n] = c \cdot \lambda^n$

$c \cdot \lambda^{n+2} - 9 \cdot c \cdot \lambda^{n+1} + 18 \cdot c \cdot \lambda^{n}$

$\lambda^2 - 9 \cdot \lambda + 18 = 0$

$\lambda_1 = 3, \ \ \lambda_2 = 6$

$Z^{o}[n] = c_1 \cdot 3^n + c_2 \cdot 6^n$

$Z^{\text{н}}[n] = A \cdot n + B$

$A \cdot (n + 2) + B - 9 \cdot (A \cdot (n + 1) + B) + 18 \cdot (A \cdot n + B) = n + 2$

$10 \cdot A \cdot n - 7 \cdot A + 10 \cdot B = n + 2$

\underline{$A = 0.1$}

$n - 0.7 + 10 \cdot B = n + 2$

\underline{$B = 0.27$}\\[0.5cm]

$Z[n] = c_1 \cdot 3^n + c_2 \cdot 6^n +0.1 \cdot n + 0.27$\\[0.5cm]

$
 \begin{cases}
   Z(0) = 1
   \\
   Z(1) = 3
 \end{cases}
$
$
 \begin{cases}
  c_1 + c_2 + 0.27 = 1
   \\
   3 \cdot c_1 + 6 \cdot c_2 + 0.1 + 0.27 = 3
 \end{cases}
$
$
 \begin{cases}
  c_1 = \frac{7}{12}
   \\
   c_2 = \frac{11}{75}
 \end{cases}
$\\[0.5cm]

$Z[n] = \frac{7}{12} \cdot 3^n + \frac{11}{75} \cdot 6^n + 0.1 \cdot n + 0.27$\\[0.5cm]

Проверка:

С одной стороны,

 $Z[2] =  \frac{7}{12} \cdot 9 + \frac{11}{75} \cdot 36 + 0.2 + 0.27 = 11$

C другой,

 $Z[n+2] = n + 2 + 9 \cdot Z[n+1] - 18 \cdot Z[n] \Rightarrow \\ Z[2] = 2 + 9 \cdot Z[1] - 18 \cdot Z[0] = 2 + 9 \cdot 3 - 18 \cdot 1 = 11$
 

\subsection{Методом вариаций}

\end{document}
