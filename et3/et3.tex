\documentclass[a4paper,14pt]{extarticle}
\usepackage[utf8x]{inputenc}
\usepackage[T1,T2A]{fontenc}
\usepackage[russian]{babel}
\usepackage{hyperref}
\usepackage{indentfirst}
\usepackage{listings}
\usepackage{color}
\usepackage{here}
\usepackage{array}
\usepackage{multirow}
\usepackage{graphicx}
\usepackage{caption}
\usepackage{subcaption}
\usepackage{chngcntr}
\usepackage{amsmath}
\usepackage{pgfplots}
\usepackage{pgfplotstable}
\counterwithin{figure}{section}
\counterwithin{equation}{section}
\counterwithin{table}{section}
\usepackage{tabularx}

%% Поля подписи и даты
\newcommand{\sign}[1][5cm]{%
\makebox[#1]{\hrulefill}
}

\usepackage[left=2cm,right=2cm,
top=2cm,bottom=2cm,bindingoffset=0cm]{geometry}


\begin{document}	% начало документа

\begin{titlepage}	% начало титульной страницы

	\begin{center}		% выравнивание по центру

		\large Санкт-Петербургский Политехнический Университет Петра Великого\\
		\large Институт компьютерных наук и технологий \\
		\large Кафедра компьютерных систем и программных технологий\\[4cm]
		% название института, затем отступ 6см
		
		 \huge Вычислительная математика\\[0.3cm] % название работы, затем отступ 0,5см
		 \large Расчётное задание №4\\[0.1cm]
		 \large Решение систем дифференциальных и разностных уравнений II порядка\\[8cm]

	\end{center}


	\begin{flushright} % выравнивание по правому краю
		\begin{minipage}{0.35\textwidth} % врезка в половину ширины текста
			\begin{flushleft} % выровнять её содержимое по левому краю

				\large\textbf{Работу выполнил:}\\
				\large Ламтев А.Ю.\\
				\large {Группа:} 23501/4\\
				
				\large \textbf{Преподаватель:}\\
				\large Цыган В.Н.

			\end{flushleft}
		\end{minipage}
	\end{flushright}
	
	\vfill % заполнить всё доступное ниже пространство

	\begin{center}
	\large Санкт-Петербург\\
	\large \the\year % вывести дату
	\end{center} % закончить выравнивание по центру

\thispagestyle{empty} % не нумеровать страницу
\end{titlepage} % конец титульной страницы

\vfill % заполнить всё доступное ниже пространство



\section{Задание}

\begin{displaymath}
I = \int_{0}^{1} (\sqrt{1 + 2 \cdot x} + x)\ \ dx
\end{displaymath}

Вычислить приближённое значение интеграла I

\begin{enumerate}

\item По формуле Симпсона двумя способами:

\begin{itemize}

\item применив формулу ко всему интервалу интегрирования

\item применив формулу к двум половинам интервала.

\end{itemize}

\item По формуле Чебышева с тремя узлами

\item По формуле Гаусса с тремя узлами

\item В целях исследования погрешностей интегрирования сравнить между собой приближённые значения интеграла, полученные всеми четырьмя способами. Вычислить реальные погрешности интегрирования \\ $\varepsilon = I_\text{точн.} - I_\text{прибл.}$ для всех четырёх способов.

\end{enumerate}

\section{Решение}

Значение определенного интеграла с точностью до 9-го знака после запятой

$I = \int_{0}^{1} (\sqrt{1 + 2 \cdot x} + x)\ \ dx = \Big ( \frac{\sqrt{(1 + 2 \cdot x)^3}}{3} + \frac{x^2}{2} \Big ) \Big |_0^1 = \sqrt{3} + \frac{1}{6} \simeq 1.898717474$

\subsection{По формуле Симпсона}

\begin{displaymath}
I = \frac{b - a}{6} \cdot \Big ( f(a) + 4 \cdot f \Big (\frac{b+a}{2} \Big ) + f(b) \Big )
\end{displaymath}

\subsubsection{Весь интервал}

$I = \frac{1 - 0}{6} \cdot \Big ( 1 + 4 \cdot (\sqrt{2} + 0.5) + \sqrt{3} + 1 \Big ) = \frac{1}{6} \cdot \Big ( 1 + 4 \cdot (1.414213562 + 0.5) + 1.732050807 + 1 \Big ) = 1.898150842$

\subsubsection{Две половины интервала}

\begin{displaymath}
I_1 = \int_{0}^{0.5} (\sqrt{1 + 2 \cdot x} + x)\ \ dx
\end{displaymath}

$I_1 = \frac{0.5 - 0}{6} \cdot \Big ( 1 + 4 \cdot (\sqrt{1.5} + 0.25) + \sqrt{2} + 0.5 \Big ) = \frac{1}{12} \cdot \Big ( 1 + 4 \cdot (1.224744871 + 0.25) + 1.414213562 + 0.5 \Big ) = 0.734432754$

\begin{displaymath}
I_2 = \int_{0.5}^{1} (\sqrt{1 + 2 \cdot x} + x)\ \ dx
\end{displaymath}

$I_2 = \frac{1 - 0.5}{6} \cdot \Big ( \sqrt{2} + 0.5 +  4 \cdot (\sqrt{2.5} + 0.75) + \sqrt{3} + 1 \Big ) = \frac{1}{12} \cdot \Big ( 1.414213562 + 0.5 +  9.324555320 + 1.732050807 + 1 \Big ) = 1.164234974$

$I = I_1 + I_2 = 1.898667728$

\subsection{По формуле Чебышева с тремя узлами}

\begin{displaymath}
I = A \sum_{k=1}^{s} f(x_k) = \frac{b - a}{s} \sum_{k=1}^{s} f(x_k),\ \ s = 3
\end{displaymath}

$x_k = \frac{a + b}{2} + \frac{b - a}{2} \cdot t_k$


$
 \begin{cases}
   t_1 = - \frac{\sqrt{2}}{2}
   \\
   t_2 = 0
   \\
   t_3 = \frac{\sqrt{2}}{2}
 \end{cases}
$


$
 \begin{cases}
   x_1 = - \frac{2 - \sqrt{2}}{4} = 0.146446609
   \\
   x_2 = 0.500000000
   \\
   x_3 = \frac{2 + \sqrt{2}}{4} = 0.8535533905
 \end{cases}
$

$
 \begin{cases}
   f(x_1) = 1.283507606
   \\
   f(x_2) = 1.914213562
   \\
   f(x_3) = 2.498876715
 \end{cases}
$

$I = \frac{f_1 + f_2 + f_3}{3} = \frac{5.696597883}{3} = 1.898865961$

\subsection{По формуле Гаусса с тремя узлами}

\begin{displaymath}
I = A \sum_{k=1}^{s} f(x_k) = \frac{b - a}{2} \sum_{k=1}^{s} A_k \cdot f(x_k),\ \ s = 3
\end{displaymath}

$x_k = \frac{a + b}{2} + \frac{b - a}{2} \cdot t_k$


$
 \begin{cases}
   t_1 = - \sqrt{\frac{3}{5}}
   \\
   t_2 = 0
   \\
   t_3 = \sqrt{\frac{3}{5}}
 \end{cases}
$


$
 \begin{cases}
   x_1 = \frac{1}{2} \cdot ( 1 - \sqrt{\frac{3}{5}}) = 0.112701665
   \\
   x_2 = 0.500000000
   \\
   x_3 = \frac{2 + \sqrt{2}}{4} = 0.887298334
 \end{cases}
$

$
 \begin{cases}
   f(x_1) = 1.219681036
   \\
   f(x_2) = 1.914213562
   \\
   f(x_3) = 2.553010394
 \end{cases}
$

$
 \begin{cases}
   A_1 = A_3 = \frac{5}{9}
   \\
   A_2 = \frac{8}{9}
 \end{cases}
$

\begin{displaymath}
I = \frac{1}{2} \sum_{k=1}^{s} A_k \cdot f(x_k) = \frac{1}{18} \cdot (5 \cdot (1.219681036 + 2.553010394) + 8 \cdot 1.914213562)
\end{displaymath}

$ = 1.898731424$

\subsection{Исследование погрешностей вычислений}

\begin{displaymath}
\varepsilon = I_\text{точн.} - I_\text{прибл.}
\end{displaymath}

\begin{itemize}

\item Формула Симпсона для всего интервала:

$\varepsilon = 1.898717474 - 1.898150842 = 0.000566631$

\item Формула Симпсона для двух половин интервала

$\varepsilon = 1.898717474 - 1.898667728 = 0.000049745$

\item Формула Чебышева с тремя узлами

$\varepsilon = 1.898717474 - 1.898865961 = -0.000148487$

\item Формула Гаусса с тремя узлами

$\varepsilon = 1.898717474 - 1.898731424 = -0.000013950$

\end{itemize}

Исходя из погрешностей, можно сделать вывод, что формула Гаусса с тремя узлами больше остальных подходит для вычисления данного определённого интеграла, а формула Симпсона для всего интервала меньше других формул подходит для вычисления данного определённого интеграла. 

\end{document}
